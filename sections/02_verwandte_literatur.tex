\section{Verwandte Literatur}
\label{sec:verwandte-literatur}

Die Arbeit bezieht sich größtenteils auf die Fachschriften der \textit{Moral Machine Website}. Unterschiedliche Grundlagenliteratur beschäftigt sich mit allgemeinen Begriffen und Technologien des autonomen Fahrens.\\

\citeauthor{roadblocks} identifizieren in ihrer wissenschaftlichen Publikation \textit{Psychological roadblocks to the adoption of self-driving vehicles} \cite{roadblocks} unterschiedliche Dilemmas und auftretende Herausforderungen, deren sich die Gesellschaft stellen muss. Entscheidungsträger werden Richtlinien für diese Dilemmas finden müssen die gleichermaßen von der Gesellschaft akzeptiert werden müssen. Hierfür werden mögliche Lösungsvorschläge unterbreitet. 

Es wird identifiziert, dass die Mehrheit ein Verhalten von autonomen Fahrzeugen in Extremsituationen das dem Gesamtwohl zu gute kommt bevorzugen würde. Kaufen würden Personen jedoch Fahrzeuge die Ihnen selbst möglichst hohe Sicherheit gewährleisten. Diese beiden Fakten schließen sich gegenseitig aus.

Unausweichliche Unfälle autonomer Fahrzeuge werden zu Überreaktionen in der Gesellschaft führen, was die Einführung und Annahme verlangsamen oder lahm legen kann.

Durch die Verwendung von Anwendungen aus dem Bereich des Maschinellen Lernens, kann der Entscheidungsprozess, den autonome Fahrzeuge in bestimmten Situationen treffen, schlecht nachvollzogen werden. Das kann dazu führen, dass ein allgemeines  Misstrauen gegenüber der Technologie  entsteht.\\

In \textit{The Social Dilemma of Autonomous Vehicles} \cite{socialDilemma} beschreiben \citeauthor{socialDilemma} die Ergebnisse eine Studie mit 1929 Teilnehmern. Diese wurden dazu befragt sie Fahrzeuge ob Sie mit einem zweckorientierten Verhalten in Extremsituationen bevorzugen und ob sie diese kaufen würden. Damit baut die Publikation auf dem bereits identifizierten ethischen Dilemma aus \cite{roadblocks} auf. Über Auswertungen wird gezeigt, dass eine Regulierung von autonomen Fahrzeugen durch den Gesetzgeber zu einem zweckorientierten Verhalten die Einführung verlangsamen würden. Das hätte in der Gesamtheit mehr Todesfälle zur Folge als die frühe Einführung der Technologie, da die langfristig mehr Unfälle durch Menschen verursacht werden als durch autonome Fahrzeuge. 90\% der Unfälle sind auf menschliche Fehler zurückzuführen.\\

Der Artikel \textit{The Moral Machine experiment} \cite{moralMachine} von \citeauthor{moralMachine} untersucht die Daten, die über die eine weltweit verfügbare Umfrageplattform \textit{The Moral Machine} gesammelt wurden. Nutzer wurden nach dem moralisch besser vertretbaren Ausgang verschiedener unausweichlicher Unfallszenarien befragt. Zur Auswahl hat ein Nutzer jeweils zwei unterschiedliche Ausgänge in eine Extremsituationen. Bei den Ausgängen müssen entweder die Insassen des Fahrzeugs, oder Passenten auf der Straße sterben. In den Szenarien werden in auf der Straße und im Fahrzeug jeweils unterschiedliche Personengruppen abgebildet. Jedes Unfallszenarium stellt dabei ein ethisches Dilemma dar. Es wurden unterschiedliche Statistiken erhoben auf die in dieser Arbeit referenziert wird. Außerdem wurden Beobachtungen gemacht und beispielsweise ein Unterschied in der moralischen Bewertung der Unfallszenarien in verschiedenen Kulturkreisen identifiziert. Es wird dabei wesentlich zwischen Westlichem, Südlichem und Östlichem Kulturkreis unterschieden. Gesamtheitlich wurde beobachtet, dass es starke Präferenzen der Befragten zum \textit{Retten von Menschen (gegenüber Tieren)}, zum \textit{Retten von vielen Leben (gegenüber weniger Leben)} und zum \textit{Retten von jungen Menschen (gegenüber alten Menschen)} gibt. Zur Zeitpunkt der Erstellung des Artikels, \citeyear{moralMachine}, wurden 39,61 Millionen Entscheidungen aus 233 Ländern weltweit gesammelt. \\

\textit{A Voting-Based System for Ethical Decision Making}, \cite{votingBasedSystem}, verfasst von \citeauthor{votingBasedSystem}, schlägt einen Algorithmen vor mit dem ethische Entscheidungen in Grenzsituationen getroffen werden können. Es handelt sich dabei auf einen technischen Algorithmus, der anhand der Daten die mit \textit{The Moral Machine} gesammelt wurde Entscheidungen treffen kann. Der Algorithmus wertet die Ergebnisse aller Teilnehmern der Plattform aus, führt diese zusammen und ist in der Lage  \textit{swap-dominance efficient} das finden von Alternativen zu erlernen. Methoden des Maschinellen Lernens werden verwendet und der trainierte Algorithmus ist in der Lage, die ethisch am besten zu akzeptierenden Ausgänge für eine Extremsituationen aus einer Menge von Alternativen zu finden.\\

Neben der Literatur von \textit{The Moral Machine} baut dieses Arbeit auf unterschiedlichen Grundlagewerken zum Thema des autonomen Fahrens auf. Der vom \citeauthor{smith2015automated} veröffentlichte Bericht \textit{Automated and
autonomous driving: regulation under uncertainty} \cite{smith2015automated}. Dieser beschreibt allgemeine Technologien und Terminologien der Domäne. Außerdem werden die SAE-Levels, die in im Standard \textit{Taxonomy and Definitions for Terms Related to Driving Automation Systems for On-Road Motor Vehicles} \cite{standardSAE} der \citeauthor{standardSAE}, vorgestellt. Zur Vollständigkeit werden diese auch in dieser Arbeit in \ref{ssec:sae-level} beschrieben.\\