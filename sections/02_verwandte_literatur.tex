\section{Verwandte Literatur}
\label{sec:verwandte-literatur}

Die Arbeit bezieht sich auf die Fachschriften der Internetseite \textit{Moral Machine Website}. Des weiteren wird unterschiedliche Grundlagenliteratur, die sich mit allgemeinen Begriffen und Technologien des autonomen Fahrens beschäftigt, vorgestellt.\\

\citeauthor{roadblocks} identifizieren in ihrer wissenschaftlichen Publikation \textbf{Psychological roadblocks to the adoption of self-driving vehicles \cite{roadblocks}} unterschiedliche Dilemmas und auftretende Herausforderungen, derer sich die Gesellschaft stellen muss. Entscheidungsträger werden also zukünftig Richtlinien für diese komplexen Konflikte erarbeiten müssen, die wiederum Zuspruch in der Gesellschaft finden müssen. Hierfür werden mögliche Lösungsansätze geboten. 

Es wurde festgestellt, dass die Mehrheit ein Verhalten bevorzugt, in dem  autonome Fahrzeuge in Extremsituationen im Sinne des größtmöglichen Allgemeinwohls handeln würden. Kaufen würden diese Personen jedoch nur Fahrzeuge, die Ihnen selbst ein möglichst hohes Maß an Sicherheit bieten. Diese beiden Faktoren schließen sich aber gegenseitig aus.

Unausweichliche Unfälle autonomer Fahrzeuge könnten zu Überreaktionen in der Gesellschaft führen, was die Einführung und Akzeptanz verlangsamen oder lahm legen könnte.

Durch die Verwendung von Anwendungen aus dem Bereich des Maschinellen Lernens kann der Entscheidungsprozess, den autonome Fahrzeuge in bestimmten Situationen treffen, schlecht nachvollzogen werden. Das kann dazu führen, dass ein allgemeines Misstrauen gegenüber der Technologie entsteht.\\

In \textbf{The Social Dilemma of Autonomous Vehicles \cite{socialDilemma}} beschreiben \citeauthor{socialDilemma} die Ergebnisse einer Studie mit 1929 Teilnehmern. Diese wurden dazu befragt, ob sie Fahrzeuge mit einem nutzenorientierten Verhalten in Extremsituationen bevorzugen und ob sie diese kaufen würden. Damit baut die Publikation auf den bereits voran beschriebenen ethischen Kontroversen \cite{roadblocks} auf. Anhand der Auswertungen wird gezeigt, dass eine Regulierung autonomer Fahrzeuge, denen ein nutzenorientiertes Verhalten zugrunde liegt, durch den Gesetzgeber die Einführung verlangsamen würde. Das hätte in der Gesamtheit mehr Todesfälle zur Folge als die frühe Einführung der Technologie, da langfristig mehr Unfälle durch Menschen verursacht werden als durch autonome Fahrzeuge. Derzeit sind 90\% der Unfälle auf menschliches Fehlverhalten zurückzuführen.\\

Der Artikel \textbf{The Moral Machine experiment \cite{moralMachine}} von \citeauthor{moralMachine} untersucht die Daten, die über die eine weltweit verfügbare Umfrageplattform \textit{The Moral Machine} gesammelt wurden. Nutzer wurden nach dem moralisch besser vertretbaren Ausgang verschiedener unausweichlicher Unfallszenarien befragt. Zur Auswahl konnten die Befragten jeweils zwischen zwei unterschiedlichen Ausgängen in einer Notlage wählen. Bei den Ausgängen müssen entweder die Insassen des Fahrzeugs oder Passanten auf der Straße sterben. In den Szenarien werden auf der Straße und im Fahrzeug jeweils unterschiedliche Personengruppen abgebildet. Jedes Unfallszenarium stellt dabei eine Crux dar. Es wurden unterschiedliche Statistiken erhoben, die in dieser Arbeit referenziert werden. Außerdem wurde ein Unterschied in der moralischen Bewertung der Unfallszenarien in verschiedenen Kulturkreisen beobachtet. Es wird dabei im Wesentlichen zwischen westlichem, südlichem und östlichem Kulturkreis unterschieden. Gesamtheitlich wurde festgestellt, dass es starke Präferenzen der Befragten zum \textit{Retten von Menschen (gegenüber Tieren)}, zum \textit{Retten von vielen Leben (gegenüber weniger Leben)} und zum \textit{Retten von jungen Menschen (gegenüber alten Menschen)} gibt. Zum Zeitpunkt der Erstellung des Artikels, \citeyear{moralMachine}, wurden 39,61 Millionen Entscheidungen aus 233 Ländern weltweit gesammelt. \\

\textbf{A Voting-Based System for Ethical Decision Making \cite{votingBasedSystem}}, verfasst von \citeauthor{votingBasedSystem}, schlägt einen Algorithmus, mit dem ethische Problematiken in Miseren getroffen werden können, vor. Es handelt sich dabei um einen technischen Algorithmus, der anhand der Daten, die von \textit{The Moral Machine} gesammelt wurden, Entscheidungen treffen kann. Der Algorithmus wertet die Ergebnisse aller Teilnehmenden der Plattform aus, führt diese zusammen und ist in der Lage  \textit{swap-dominance efficient} das Finden von Alternativen zu erlernen. Methoden des Maschinellen Lernens werden verwendet und der trainierte Algorithmus ist in der Lage, die ethisch am besten zu akzeptierenden Ausgänge für eine Extremsituationen aus einer Menge von Optionen zu wählen.\\

Neben der Literatur von \textit{The Moral Machine} baut dieses Arbeit auf unterschiedlichen Grundlagewerken zum Thema des autonomen Fahrens auf. Der vom \citeauthor{smith2015automated} veröffentlichte Bericht \textbf{Automated and
autonomous driving: regulation under uncertainty \cite{smith2015automated}} beschreibt allgemeine Technologien und Terminologien der Domäne.\\

Außerdem werden die SAE-Level des Standard \textbf{Taxonomy and Definitions for Terms Related to Driving Automation Systems for On-Road Motor Vehicles \cite{standardSAE}} der \citeauthor{standardSAE} vorgestellt (siehe. \ref{ssec:sae-level}).\\