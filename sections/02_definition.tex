\section{Definition des ITF}
\cite{websiteITF} Das ITF (International Transport Forum) ist ein unter dem OECD (Organisation for Economic Co-operation and Development) zwischenstaatliche Organisation die es sich zum Ziel gesetzt hat, ein tieferes Verständnis über das Transportwesen in Rollen des Wirtschaftswachstums, ökologischer Nachhaltigkeit und ethischen Aspekten zu fördern. Bestehend aus 60 Mitgliedsländern agiert es als Ideenschmiede für Richtlinie und organisiert das jährliche Gipfeltreffen der Transportminister. In unregelmäßigen Abständen veröffentlicht der ITF Berichte den verschiedenen Themen des Transportwesens.

\subsection{SEA-Levels}

\cite{standardSea}\cite{smith2015automated} Das ITF stützt sich auf die von der SEA (International Society of Automotive Engineers) definierten sechs Level des autonomen Fahrens. Diese reichen von komplett automatisierten Fahrzeugen bis hin zu komplett automatisierten Transportmittel, bei denen kein Fahrer mehr notwendig ist.\\

\textbf{TODO!}

\begin{itemize}
    \item Level 0 - Keine Automatisierung: Der Fahrer übernimmt die volle Kontrolle über das Fahrzeug.
    \item Level 1 - Fahrassistenz:
    \item Level 2 - Partielle Automatisierung:
    \item Level 3 - Bedingte Automatisierung:
    \item Level 4 - Hohe Automatisierung:
    \item Level 5 - Volle Automatisierung:
\end{itemize}