\section{Definition des ITF}
\cite{websiteITF} Das ITF (International Transport Forum) ist ein unter dem OECD (Organisation for Economic Co-operation and Development) zwischenstaatliche Organisation die es sich zum Ziel gesetzt hat, ein tieferes Verständnis über das Transportwesen in Rollen des Wirtschaftswachstums, ökologischer Nachhaltigkeit und ethischen Aspekten zu fördern. Bestehend aus 60 Mitgliedsländern agiert es als Ideenschmiede für Richtlinie und organisiert das jährliche Gipfeltreffen der Transportminister. In unregelmäßigen Abständen veröffentlicht der ITF Berichte den verschiedenen Themen des Transportwesens.

\subsection{SAE-Levels}

\cite{standardSAE}\cite{smith2015automated} Das ITF stützt sich auf die von der SAE (International Society of Automotive Engineers) definierten sechs Level des autonomen Fahrens. Diese reichen von komplett automatisierten Fahrzeugen bis hin zu komplett automatisierten Transportmittel, bei denen kein Fahrer mehr notwendig ist.\\

\begin{itemize}
    \item Level 0 - Keine Automatisierung: Der Fahrer übernimmt die volle Kontrolle über das Fahrzeug.\\

    \item Level 1 - Fahrassistenz: Der Fahrer wird durch einen einzelnen Fahrassistenten, z.B. Tempomat, Spurhalteassistent oder Bremsassistent, unterstützt, behält jedoch die volle Kontrolle über das Fahrzeug.\\
    
    \item Level 2 - Partielle Automatisierung: Der Fahrer wird durch mehrere Fahrassistenten unterstützt and hat immer die Kontrolle über das Fahrzeug.\\
    
    \item Level 3 - Bedingte Automatisierung: Das Fahrzeug ist in der Lage die Umgebung zu erkennen und ist in der Lage eigenständig in bestimmten Fahrsituationen zu meistern. Ein aufmerksamer Fahrer ist noch erforderlich, der unter Umständen einschreiten kann.\\
    
    \item Level 4 - Hohe Automatisierung: Das Fahrzeug ist in der Lage sich eigenständig in den meisten Fahrumgebungen zu manövrieren. Es erkennt Fehlentscheidungen und kann auf diese reagieren. Menschliches einschreiten ist, wenn gewünscht, möglich.\\
    
    \item Level 5 - Volle Automatisierung: Das Fahrzeug kann in allen Umgebungen und zu jederzeit und in jeder Situation komplett eigenständig Fahren.\\
\end{itemize}