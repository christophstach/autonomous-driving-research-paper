\section{Akteure}
\label{sec:akteure}

Im Feld des autonomen Fahren tummeln such unterschiedliche Akteure. Einige nehmen passiv daran in dem Sie Richtlinie erstellen und Diskussionen anregen. Andere treiben aktiv die Weiterentwicklung in dem Bereich voran.

\subsubsection*{International Transport Forum} Das International Transport Forum (ITF)\footnote{\url{https://www.itf-oecd.org/}} ist ein unter dem OECD (Organisation for Economic Co-operation and Development) zwischenstaatliche Organisation die es sich zum Ziel gesetzt hat, ein tieferes Verständnis über das Transportwesen in Rollen des Wirtschaftswachstums, ökologischer Nachhaltigkeit und ethischen Aspekten zu fördern. Bestehend aus 60 Mitgliedsländern agiert es als Ideenschmiede für Richtlinie und organisiert das jährliche Gipfeltreffen der Transportminister. In unregelmäßigen Abständen veröffentlicht der ITF Berichte den verschiedenen Themen des Transportwesens.

\subsubsection*{Society of Automotive Engineers \cite{standardSAE}\cite{smith2015automated}} Das ITF stützt sich auf die von der Society of Automotive Engineers (SAE)\footnote{\url{https://www.sae.org/}}  definierten sechs Level des autonomen Fahrens. Diese reichen von komplett automatisierten Fahrzeugen bis hin zu komplett automatisierten Transportmittel, bei denen kein Fahrer mehr notwendig ist.\\