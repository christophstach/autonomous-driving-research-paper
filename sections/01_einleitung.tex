\section{Einleitung}

Heutzutage wird der Großteil der Fahrzeuge auf unseren Straßen durch Menschen gesteuert. Doch das könnte sich bald ändern. Schon jetzt unterstützen Fahrassistenzsysteme die Fahrer und warnen unter Umständen in Gefahrensituationen.

Das Forschungsfeld des autonomen Fahrens besteht aus unterschiedliche Teilgebiete. In diesem Artikel wird  zwischen dem rein technischen Bereich und dem sozial-technischen Bereich unterschieden. Technologischen Forschungen beziehen sich auf Hardware, z.B. unterschiedliche Sensoren, und Software, z.B. Algorithmen und Assistenzsysteme, die benötigt werden um ein autonom fahrendes Fahrzeug zu realisieren. Der sozial-technische Bereich umfasst die Auswirkungen auf die Gesellschaft und das Vertrauen in die neuartige Technologie. Die Aspekte die durch autonomes Fahren entstehen werden Versucht durch Gesetzgeber mit Richtlinien zu regulieren und Verfahren zu standardisieren. Dadurch entsteht zur Zeit ein internationaler Wettlauf um die Vorherrschaft in diesem Gebiet.

\subsection{Verwandte Literatur}

\textit{Psychological roadblocks to the adoption of self-driving vehicles}  \cite{roadblocks} identifiziert unterschiedliche Dilemmas und auftretende Herausforderungen, deren sich die Gesellschaft annehmen muss und die sich der Gesellschaft und den Entscheidungsträgern stellen werden müssen. Außerdem werden Lösungsvorschläge unterbreitet.


\begin{itemize}
    \item Das Volk möchte das sich autonome Fahrzeuge in extrem Situationen so verhalten, dass es dem Gesamtwohl zu gute kommt. Kaufen würden Personen jedoch Fahrzeuge die Ihnen selbst möglichst hohe Sicherheit gewährleisten.
    \item Unausweichliche Unfälle werden zu Überreaktionen der Gesellschaft, welche durch die Medien publiziert werden, führen. Das kann die flächendeckende Einführung von autonomen Fahrzeugen verlangsamen oder sogar stilllegen.
    \item Durch die Verwendung von Anwendungen aus dem Bereich des Maschinellen Lernens, kann der Entscheidungsprozess den autonome Fahrzeuge in bestimmten Situationen treffen, schlecht nachvollzogen werden. Das kann dazu führen das ein Misstrauen gegenüber der Technologie in der Gesellschaft entsteht.
\end{itemize}


In \textit{The Social Dilemma of Autonomous Vehicles} \cite{socialDilemma} bla bla bla bla



Der Artikel \textit{The Moral Machine experiment} \cite{moralMachine} untersucht die Daten die über die eine weltweit verfügbare Umfrageplattform\footnote{\url{http://moralmachine.mit.edu/}} gesammelt worden sind. Nutzer wurden nach dem moralisch besser vertretbaren Ausgang verschiedener unausweichlicher Unfallszenarien befragt. Dabei wurde unterschiedliche Statistiken erhoben auf die in dieser Arbeit referenziert wird. Außerdem wurden Beobachtungen gemacht und beispielsweise ein Unterschied in der moralischen Bewertung der Unfallszenarien in verschiedenen Kulturkreisen identifiziert. Es wird dabei wesentlich zwischen Westlichem, Südlichem und Östlichem Kulturkreis unterschieden.


\textit{A Voting-Based System for Ethical Decision Making} \cite{votingBasedSystem} schlägt einen Algorithmen vor mit dem ethische Entscheidungen in Grenzsituationen getroffen werden können.