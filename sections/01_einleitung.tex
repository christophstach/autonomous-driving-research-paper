\section{Einleitung}

Heutzutage wird der Großteil der Fahrzeuge auf unseren Straßen durch Menschen gesteuert. Doch das könnte sich bald ändern. Schon jetzt unterstützen Fahrassistenzsysteme die Fahrer und warnen unter Umständen in Gefahrensituationen.\\

Das Forschungsfeld des autonomen Fahrens besteht aus unterschiedliche Teilgebiete. In diesem Artikel wird  zwischen dem rein technischen Bereich und dem sozial-technischen Bereich unterschieden. Technologischen Forschungen beziehen sich auf Hardware, z.B. unterschiedliche Sensoren, und Software, z.B. Algorithmen und Assistenzsysteme, die benötigt werden um ein autonom fahrendes Fahrzeug zu realisieren. Der sozial-technische Bereich umfasst die Auswirkungen auf die Gesellschaft und das Vertrauen in die neuartige Technologie. Die Aspekte die durch autonomes Fahren entstehen werden Versucht durch Gesetzgeber mit Richtlinien zu regulieren und Verfahren zu standardisieren. Dadurch entsteht zur Zeit ein internationaler Wettlauf um die Vorherrschaft in diesem Gebiet.\\

\subsection{Verwandte Literatur}

Test