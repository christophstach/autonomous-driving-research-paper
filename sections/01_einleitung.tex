\section{Einleitung}

Aktuell wird noch ein Großteil der Fahrzeuge auf Straßen und im öffentlichen Verkehr durch Menschen gesteuert. Doch das könnte sich bald ändern. Schon jetzt unterstützen Fahrassistenzsysteme die Fahrer und warnen in Gefahrensituationen.\\

Das Forschungsfeld des autonomen Fahrens besteht aus unterschiedlichen Teilgebieten. In diesem Artikel wird zwischen dem rein technischen Bereich und dem sozialtechnischen Sektor unterschieden. Technologische Forschungen beziehen sich auf Hardware (z. B. unterschiedliche Sensoren) und Software (z. B. Algorithmen und Assistenzsysteme), die benötigt werden, um autonomes Fahrzeug zu realisieren. Der sozialtechnische Bereich umfasst die Auswirkungen auf die Gesellschaft und das Vertrauen in die neuartige Technologie. Die Herausforderungen, die dadurch entstehen, werden versucht durch Gesetzgeber mit Richtlinien zu regulieren, um Verfahren zu standardisieren. Dadurch entsteht zur Zeit ein internationaler Wettlauf um die Vorherrschaft in diesem Gebiet.\\

In dieser Arbeit geht der Autor in \ref{sec:verwandte-literatur} auf die mit dem Thema verwandte Arbeiten ein. Die unterschiedlichen wissenschaftlichen Publikationen, die sich mit der Internetseite \textit{The Moral Machine}\footnote{\url{http://moralmachine.mit.edu/}} beschäftigen, werden vorgestellt. Danach wird sich mit den wesentlichen Definitionen und Technologien in \ref{sec:definitionen-und-technologie}, die in autonomen Fahrzeugen zum Einsatz kommen, beschäftigt. Schließlich wird eine ethische Diskussion in \ref{sec:diskussion} angeregt und der Autor begründet seine persönlichen Gedankengänge in einem Fazit, vgl. \ref{sec:fazit}.