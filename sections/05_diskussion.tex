\section{Diskussion}
\label{sec:diskussion}

Die Arbeit behandelt Fahrzeuge des SAE-Levels 4 und 5. Denn nur diese Fahrzeuge sind in der Lage ohne Supervision eines Fahrers sich im Straßenverkehr zu bewegen. 

\subsection{Vorteile}

Autonomes Fahren bringt viele Vorteile mit sich. Durch die unterschiedlichen Sensoren kann weitaus größeres Sichtfeld erzeugt werden, als es mit dem menschlichen Auge möglich ist. Software trifft Entscheidung binnen Millisekunden und kann akkurate Berechnungen über den Fahrtverlauf durchführen. 
Da es sich bei 90\% der Unfälle um menschliches Fehlverhalten handelt kann die durch die Einführung autonomer Fahrzeuge die Unfallrate dementsprechend gesenkt werden \cite{roadSafty}.\\

Durch die effiziente Kommunikation der Fahrzeuge untereinander können Staus vermieden werden. Der Verkehr würde sich besser auf unterschiedliche Teile des Verkehrsnetzes verteilen. Das verkürzt die Fahrzeiten insgesamt. \\

Mit geteilten autonomen Fahrzeugen, die teil der öffentlich Verkehrsmittel werden, könnte die Anzahl der Fahrzeuge auf den Straßen insgesamt verringert werden. Die meiste Zeit stehen private Fahrzeuge auf Parkplätzen. Die Parkplatzsuche würde entfallen, was gerade in Großstädten ein bekanntes Problem ist. Eine Studie der Berryls Strategy Advisor zeigt das der Einsatz von 18.000 Robotertaxis den gesamten privaten Verkehr sowie 20\% des Pendlerverkehrs der Stadt München abdecken kann \cite{advisors2017simulation}.\\

Verkürzte Fahrtzeiten und weniger Fahrzeuge auf den Straßen wirkt sich auch positiv auf die Treibstoffemissionen aus. Somit würde der Einsatz von autonomen Fahrzeugen langfristig einen Beitrag zum Umweltschutz leisten.\\

\subsection{Herausforderung}

Die flächendeckende Einführungen von autonomen Fahrzeugen wird die Gesellschaft und auch Gesetzgeber vor Herausforderungen stellen. Zur Zeit existieren nur vage Richtlinien für das Testen und den Betrieb \cite{doi:10.1080/01441647.2018.1494640}. Fraglich ist auch wie sich autonome Fahrzeuge verhalten sollen wenn Sie Staatengrenzen überqueren. Gerade in Europa sind die meisten Grenzen offen und viele Menschen pendeln zwischen Ländern. Unterschiedliche Gesetzgebungen müssen bei der Entwicklung autonomer Fahrzeuge beachtet werden.\\

Gerade in Deutschland, wo ein Auto als Statussymbol gilt, wird sich die Einführung autonomer Fahrzeuge im privaten Sektor als schwierig erweisen. Außerdem sind Deutsche sehr misstrauisch gegenüber neuer Technologie [TODO Zitat von Studie]. Bis ein ein selbstfahrendes Auto von der Mehrheit der Bevölkerung akzeptiert wird es eine Zeit dauern. Ähnliche Hürden wurden auch in der Publikation \cite{roadblocks} von \citeauthor{roadblocks} identifiziert.\\

Ein besonderes Augenmerk muss auf die Entscheidungsfindung in Extremsituation gelegt werden. Als Extremsituation wird eine Situation definiert in denen ein Unfall unausweichlich ist. Beispielsweise wäre das bei einem plötzlichen mechanischen Versagen der Bremsen der Fall. Fahrzeuge müssen in diesen Situation den akzeptabelsten Ausgang finden. In der nächsten Sektion \ref{ssec:entscheidungen-in-extremsituationen} wird gesondert diesen Spezialfall eingegangen. Die Studie \cite{socialDilemma} geht gesondert auf diese Thematik ein. Sie zeigt, dass unterschiedliches Implementierungsmethoden für die Entscheidungsfindung in Extremsituationen das Kaufverhalten und somit die Zeit bis autonome Fahrzeuge gesellschaftlich akzeptiert werden, beeinflussen. Das würde sich besonders bei einem technikskepsischen Volk wie den Deutschen negativ auf die Akzeptanzzeit auswirken. In \citeauthor{socialDilemma} wird ebenfalls zeigt das die Regulierung durch Gesetzgeber diesen Prozess noch weiter verlangsamen kann.\\


\subsection{Entscheidungen in Extremsituationen}
\label{ssec:entscheidungen-in-extremsituationen}

Wie werden Entscheidungen autonomer Autos in Extremsituationen getroffen? Diese Frage beschäftigt Ethikwissenschaftler. Über die Internetseite \textit{The Moral Machine} wurden weltweit große Mengen von Daten erhoben. Besucher der Internetseite werden befragt welcher der ethisch besser zu akzeptierende Ausgang einer Extremsituation ist. Dabei muss der Teilnehmer der Befragung jeweils zwischen Ausgängen wählen. Man hat die Wahl das auto in eine Barriere fahren zu lassen und somit die Insassen zu töten oder Menschen die rechtens oder auch unrechtens die Ampel überqueren zu töten. Die gesammelten Daten stehen frei zur Verfügung. 

