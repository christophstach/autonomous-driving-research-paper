\section{Diskussion}
\label{sec:diskussion}

Die Arbeit behandelt Fahrzeuge des SAE-Levels 4 und 5. Denn nur diese Fahrzeuge sind in der Lage ohne Supervision eines Fahrers sich im Straßenverkehr zu bewegen. 

\subsection{Vorteile}

Autonomes Fahren bringt viele Vorteile mit sich. Durch die unterschiedlichen Sensoren kann weitaus größeres Sichtfeld erzeugt werden, als es mit dem menschlichen Auge möglich ist. Software trifft Entscheidung binnen Millisekunden und kann akkurate Berechnungen über den Fahrtverlauf durchführen. 
Da es sich bei 90\% der Unfälle um menschliches Fehlverhalten handelt kann die durch die Einführung autonomer Fahrzeuge die Unfallrate dementsprechend gesenkt werden \cite{roadSafty}.

Durch die effiziente Kommunikation der Fahrzeuge untereinander können Staus vermieden werden. Der Verkehr würde sich besser auf unterschiedliche Teile des Verkehrsnetzes verteilen. Das verkürzt die Fahrzeiten insgesamt. 

Mit geteilten autonomen Fahrzeugen, die teil der öffentlich Verkehrsmittel werden, könnte die Anzahl der Fahrzeuge auf den Straßen insgesamt verringert werden. Die meiste Zeit stehen private Fahrzeuge auf Parkplätzen. Die Parkplatzsuche würde entfallen, was gerade in Großstädten ein bekanntes Problem ist. Eine Studie der Berryls Strategy Advisor zeigt das der Einsatz von 18.000 Robotertaxis den gesamten privaten Verkehr sowie 20\% des Pendlerverkehrs der Stadt München abdecken kann \cite{advisors2017simulation}.

Verkürzte Fahrtzeiten und weniger Fahrzeuge auf den Straßen wirkt sich auch positiv auf die Treibstoffemissionen aus. Somit würde der Einsatz von autonomen Fahrzeugen langfristig einen Beitrag zum Umweltschutz leisten.




