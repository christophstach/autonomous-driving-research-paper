\begin{abstract}
    Autonomes Fahren ist ein noch relativ junges Forschungsfeld. Neue technologische Fortschritte machen es möglich, dass sich Fahrzeuge gänzlich ohne oder nur mit minimaler menschlicher Überwachung eigenständig auf Straßen oder in dafür vorgesehen Umgebungen bewegen können. Doch welche Gefahren bestehen beim autonomen Fahren und wer entscheidet in auftretenden Ernstfällen über den Ausgang einer Extremsituation? Das sind u. a. Fragen mit denen sich derzeit Wissenschaft und Gesellschaft gleichermaßen intensiv beschäftigten. Mit dieser Arbeit möchte der Author einen kurzen allgemeinen Überblick über das Themengebiet geben, z. B. welche Technologien zum Einsatz kommen, und versuchen, speziell das Treffen moralischer Entscheidungen in Extremsituation beim autonomen Fahren in der EU getroffen werden.
\end{abstract}