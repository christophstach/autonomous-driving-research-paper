\begin{abstract}
    Autonomes Fahren ist ein junges Forschungsgebiet. Neue technologische Fortschritte machen es möglich, dass sich Fahrzeuge ohne oder mit minimaler menschlicher Überwachung eigenständig auf Straßen oder in dafür vorgesehen Umgebungen bewegen können. Doch welche Gefahren bestehen beim autonomen Fahren? Welche Technologien kommen zum Einsatz und wer entscheidet in auftretenden Ernstfällen über den Ausgang einer Extremsituation. Das sind Fragen mit denen sich Wissenschaftler und die Gesellschaft beschäftigten. Mit dieser Arbeit gibt der Author einen Überblick über das Themengebiet und und versucht entstandenen Fragen zu klären. Letztlich wird eine Schlussfolgerung gezogen wie Moralische Entscheidungen in Bezug auf Extremsituation beim autonomen Fahren in der EU getroffen werden.
\end{abstract}