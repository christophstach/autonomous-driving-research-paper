

\section{Definitionen und Technologie}
\label{sec:definitionen-und-technologie}

In diesem Teil werden die grundlegenden Technologien und Definitionen beschrieben, die zum Verständnis von autonomen Fahren wichtig sind.

\subsection{SAE-Level}
\label{ssec:sae-level}

Die SAE-Level \cite{standardSAE} beschreiben den Grad an Autonomie bei Fahrzeugen. Sie werden dabei in sechs unterschiedliche Stufen eingeteilt und wurden von der \citeauthor{standardSAE} definiert.\\

\subsubsection*{Level 0 - Keine Automatisierung} Der Fahrer übernimmt die volle Kontrolle über das Fahrzeug.

\subsubsection*{Level 1 - Fahrassistenz} Der Fahrer wird durch einen einzelnen Fahrassistenten, z.B. Tempomat, Spurhalteassistent oder Bremsassistent, unterstützt, behält jedoch die volle Kontrolle über das Fahrzeug.

\subsubsection*{Level 2 - Partielle Automatisierung} Der Fahrer wird durch mehrere Fahrassistenten unterstützt and hat immer die Kontrolle über das Fahrzeug.
    
\subsubsection*{Level 3 - Bedingte Automatisierung} Das Fahrzeug ist in der Lage, die Umgebung zu erkennen und eigenständig bestimmte Fahrsituationen zu meistern. Ein aufmerksamer Fahrer ist trotzdem erforderlich, um unter Umständen einzuschreiten.

\subsubsection*{Level 4 - Hohe Automatisierung} Das Fahrzeug kann eigenständig in den meisten Fahrumgebungen manövrieren. Es erkennt Fehlentscheidungen und kann auf diese reagieren. Menschliches Einschreiten ist, wenn gewünscht, möglich.

\subsubsection*{Level 5 - Volle Automatisierung} Das Fahrzeug kann in allen Umgebungen, zu jeder Zeit und in jeder Situation komplett eigenständig fahren.

\subsection{Sensoren}

In autonomen Fahrzeugen kommen unterschiedliche Arten von Sensoren zum Einsatz. 
Dazu gehören Kameras, Radars, spezielle Laserscanner und Ultraschallsensoren.\\

\begin{figure}[H]
    \centering
    \includegraphics[width=.485\textwidth]{resources/images/sensors.png}
    \caption{Sensoren eines autonomen Fahrzeuges \cite{smith2015automated}}
\end{figure}

Kameras können rund um das Fahrzeug montiert werden und ermöglichen dem Fahrzeug somit eine vollumfängliche Sicht auf das Verkehrsgeschehen. Ein Vorteil gegenüber einem menschlichen Fahrer ist, dass auch tote Winkel mit Kameras überwacht werden können. Die Blickwinkel der Kameras lassen jedoch u. U. die Entfernung zu erkannten Objekten schlecht einschätzen.\\

Dazu wird die LIDAR-Technologie \cite{himmelsbach2008lidar} verwendet. LIDAR steht für \textit{Light Detection and Ranging}. Es handelt sich um eine Umgebungsüberwachungstechnik, die mit Hilfe eines Lasers Objekte in der Nähe bestrahlt und die Reflexion mit einem Sensor misst. So kann die Entfernung zu den Objekten berechnet werden.\\

Radar-Systeme \cite{introductionToRadarSystems} funktionieren ähnlich wie LIDAR-Systeme. Anstatt von Lichtwellen werden Radiowellen eingesetzt. Objekte in der Umgebung erzeugen ein Echo, welches von einem Sensor empfangen wird. Somit ist ein Radar-System in der Lage, die Distanz, Position und Geschwindigkeit von Objekten zu berechnen.\\

Neben Radar- und LIDAR-Systemen können weitere Sensoren in autonomen Fahrzeugen zum Einsatz kommen. Hierzu zählen beispielsweise Ultraschall und Infrarot.
Ultraschall wird auch in der Natur von Fledermäusen eingesetzt. Es funktioniert ähnlich wie ein Radar, hat jedoch eine geringere Reichweite und ist somit besser geeignet, um Objekte in der direkten Umgebung des Fahrzeuges zu erkennen.\\

Mit Hilfe der verschiedenen Sensorsystemen ist man in der Lage, ein komplettes 3D-Abbild der Umgebung zu erstellen. Damit hat ein autonomes Fahrzeug einen entscheidenden Vorteil gegenüber einem Menschen, der Objekte außerhalb seines Sichtfeldes nicht identifizieren kann.\\

\subsection{Vehicular Ad-hoc Network}

Fahrzeuge vernetzen sich mit anderen Fahrzeugen und der Infrastruktur über ein \textit{VANET}. Das \textit{VANET} ist eine Abwandlung des \textit{MANETs (Mobile Ad-hoc Network)}. 

\begin{figure}[H]
    \centering
    \includegraphics[width=.485\textwidth]{resources/images/vanet.jpg}
    \caption{Übersichtsgrafik eines VANETs \cite{vanet}}
\end{figure}

Durch die Verwendung von VANETs können Fahrzeuge untereinander und mit der Infrastruktur kommunizieren. Somit kann eine Ampel ihren aktuellen Status an das passierende Fahrzeug senden und ihm mitteilen, ob sie rot, gelb oder grün leuchtet. Das Fahrzeug kann darauf hin eine Entscheidung treffen, wie es sich verhalten soll. Eine weitere Anwendungsmöglichkeit ist beispielsweise die Stauerkennung. Fahrzeuge können sich untereinander austauschen, ob sie sich in einem Stau befinden. So können andere Verkehrsteilnehmer gewarnt und entsprechende Stauabschnitte automatisch umfahren werden.

\subsection{Software}

In autonomen Fahrzeugen kommt unterschiedliche Software zum Einsatz. Algorithmen aggregieren die Daten der Sensoren und treffen Entscheidungen über den Fahrtverlauf. Außerdem übernimmt Software die Kommunikation zu anderen Verkehrsteilnehmern und der Infrastruktur. Damit sich die Fahrzeuge besser orientieren können, werden hochauflösende Karten eingesetzt.