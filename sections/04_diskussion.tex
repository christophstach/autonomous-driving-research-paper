\section{Diskussion}
\label{sec:diskussion}

Die Arbeit behandelt Fahrzeuge des SAE-Levels 4 und 5. Denn nur diese sind in der Lage ohne Supervision eines Fahrers sich im Straßenverkehr zu bewegen. 

\subsection{Vorteile}

Autonomes Fahren bringt viele Vorteile mit sich. Durch die unterschiedlichen Sensoren kann ein weitaus größeres Sichtfeld erzeugt werden, als es mit dem menschlichen Auge möglich ist. Software trifft Entscheidung binnen Millisekunden und kann akkurate Berechnungen über den Fahrtverlauf durchführen. 
Da es sich bei 90\% der Unfälle um menschliches Fehlverhalten handelt kann die die Einführung autonomer Fahrzeuge die Unfallrate dementsprechend senken \cite{roadSafty}.\\

Durch die effiziente Kommunikation der Fahrzeuge untereinander können Staus vermieden werden. Der Verkehr würde sich besser auf unterschiedliche Teile des Verkehrsnetzes verteilen. Das verkürzt die Fahrzeiten. \\

Mit geteilten autonomen Fahrzeugen, die Teil der öffentlich Verkehrsmittel werden, kann die Anzahl der Fahrzeuge auf den Straßen verringert werden. Die meiste Zeit stehen private Fahrzeuge auf Parkplätzen. Die Parkplatzsuche würde entfallen, was gerade in Großstädten ein bekanntes Problem ist. Eine Studie der Berryls Strategy Advisor zeigt das der Einsatz von 18.000 Robotertaxis den gesamten privaten Verkehr sowie 20\% des Pendlerverkehrs der Stadt München abdecken kann \cite{advisors2017simulation}.\\

Verkürzte Fahrtzeiten und weniger Fahrzeuge auf den Straßen wirkt sich positiv auf die Treibstoffemissionen aus. Somit würde der Einsatz von autonomen Fahrzeugen langfristig einen Beitrag zum Umweltschutz leisten.\\

\subsection{Herausforderung}

Die flächendeckende Einführungen von autonomen Fahrzeugen wird die Gesellschaft und auch Gesetzgeber vor Herausforderungen stellen. Zur Zeit existieren nur vage Richtlinien für das Testen und den Betrieb \cite{doi:10.1080/01441647.2018.1494640}. Fraglich ist auch wie sich autonome Fahrzeuge verhalten sollen, wenn sie Staatengrenzen überqueren. Gerade in Europa sind die meisten Grenzen offen und viele Menschen pendeln zwischen Ländern. Unterschiedliche Gesetzgebungen müssen bei der Entwicklung autonomer Fahrzeuge beachtet werden.\\

Gerade in Deutschland, wo ein Auto als Statussymbol gilt, wird sich die Einführung autonomer Fahrzeuge im privaten Sektor als schwierig erweisen. Außerdem sind Deutsche sehr misstrauisch gegenüber neuer Technologie [TODO Zitat von Studie]. Bis ein selbstfahrendes Auto von der Mehrheit der Bevölkerung akzeptiert wird, wird es eine Zeit dauern. Ähnliche Hürden wurden auch in der Publikation \cite{roadblocks} von \citeauthor{roadblocks} identifiziert.\\

Ein besonderes Augenmerk muss auf die Entscheidungsfindung in Extremsituation gelegt werden. Als Extremsituation wird eine Situation definiert in denen ein Unfall unausweichlich ist. Beispielsweise wäre das bei einem plötzlichen mechanischen Versagen der Bremsen der Fall. Fahrzeuge müssen in diesen Situation den akzeptabelsten Ausgang finden. In der nächsten Sektion \ref{ssec:entscheidungen-in-extremsituationen} wird gesondert diesen Spezialfall eingegangen. Die Studie \cite{socialDilemma} geht gesondert auf diese Thematik ein. Sie zeigt, dass unterschiedliches Implementierungsmethoden für die Entscheidungsfindung in Extremsituationen das Kaufverhalten und somit die Zeit bis autonome Fahrzeuge gesellschaftlich akzeptiert werden, beeinflussen. Das wird sich besonders bei einem technikskepsischen Volk, wie den Deutschen, negativ auf die Akzeptanzzeit auswirken. In \citeauthor{socialDilemma} wird ebenfalls gezeigt, dass die Regulierung durch Gesetzgeber diesen Prozess noch weiter verlangsamen kann.\\


\subsection{Entscheidungen in Extremsituationen}
\label{ssec:entscheidungen-in-extremsituationen}

Wie werden Entscheidungen autonomer Autos in Extremsituationen getroffen? Diese Frage beschäftigt Ethikwissenschaftler. Über die Internetseite \textit{The Moral Machine} wurden weltweit große Mengen von Daten erhoben. Besucher der Internetseite werden befragt welcher der ethisch besser zu akzeptierende Ausgang einer Extremsituation ist. Dabei muss der Teilnehmer der Befragung jeweils zwischen zwei Ausgängen wählen. Man hat die Wahl das Auto in eine Barriere fahren zu lassen und somit die Insassen zu töten oder Menschen, die rechtens oder auch unrechtens, die Ampel überqueren zu töten. Dabei werden einem verschiedene Szenarien mit unterschiedlichen Menschen als Insassen und Passanten gezeigt. Die gesammelten Daten stehen frei zur Verfügung.\\

Mit gesammelten Daten der Internetseite wurden umfassend ausgewertet, um zu sehen wie der allgemeine Status Quo der Gesellschaft zum Thema ethische Entscheidungsfindung in Extremsituation von autonomen Fahrzeugen ist. Eine Studie hat ergeben die meisten Entscheidung zu Gunsten vieler Personen, junger Personen und Menschen (auch Tiere werden in der \textit{Moral Machine} Internetseite gezeigt) gefallen sind. Jedoch gibt es kulturelle Unterschiede. Beispielsweise werden in östlichen Kulturen, wie in vielen Asiatischen Ländern, alte Menschen bevorzugt. Es wurden drei große Hauptgruppen gefunden die, die größten Meinungsunterschiede haben. Es handelt sich um die westliche, südliche und östliche Gruppierung \cite{moralMachine}.\\

% Hier weiter lesen!

Auf der Basis der gesammelten Daten wurde außerdem ein Algorithmus \cite{votingBasedSystem} entwickelt. Dieser ist in der Lage in Extremsituation Entscheidungen zu fällen. Es ist jedoch fraglich wie die Entscheidungen gefällt werden. Europa ist ein sehr diverser Kontinent mit vielen unterschiedlichen Kulturen, Religionen und Sprachen. Die Unterschiede in der Entscheidungsfindung fallen weitaus größer aus als beispielsweise in den komplett englischsprachigen USA. Laut der Studie befindet sich Frankreich in der südlichen Gruppieren, wobei sich Deutschland in der westlichen Gruppierung befindet. Das stellt Entscheidungsträger die Herausforderung ein einheitliches System zu finden was jedem Land gerecht wird.\\

Deswegen weisen die Autoren des Algorithmus, \citeauthor{votingBasedSystem}, darauf hin das die gesammelten Daten nicht komplett sein. Befindet man sich in einer Systemsituation mit seinem Fahrzeug ergeben sich mehr Optionen als das Fahrzeug in eine Barriere fahren zu lassen oder nicht einzugreifen. Außerdem kann es sein das sich geliebte Personen oder Verwandte mit im gleichen Fahrzeug befinden. Das hätte Einfluss auf die Entscheidungsfindung bei Menschen. Das sind Faktoren die bei Sammeln der Daten mit \textit{The Moral Machine} nicht berücksichtigt worden sind.\\

Das Europäische Parlament hat im April 2019 die \textit{Ethik-Leitlinien für eine Vertrauenswürde KI} für den Umgang mit Systemen der Künstlichen Intelligenz veröffentlicht \cite{ec2019ethics}. Autonome Fahrzeuge setzen auf viele Systeme der künstlichen Intelligenz. Beispiele hier für sind die Bilderkennung und Objekterkennen oder die Algorithmen über die Entscheidungsfindung. Bei autonomen Fahrzeugen kommt es jedoch zu einem Konflikt mit diesen Leitlinien. Beispielsweise mit Leitlinie Nr. 1, \textit{Vorrang menschlichen Handelns und menschliche Aufsicht}, die bei Fahrerlosen Autos des SAE-Levels 4 oder 5 nicht zutreffen würde. Die Autoren des Dokuments, die \citeauthor{ec2019ethics}, kommentieren jedoch das die Leitlinien nicht endgültig sind. Außerdem müssen diese für bestimmte Teilgebiete der KI angepasst werden und erweitert. Das Dokument ist außerdem in Bearbeitung und soll zukünftig als Grundlage für weitere Leitlinien dienen.\\