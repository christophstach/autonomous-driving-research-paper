\section{Diskussion}
\label{sec:diskussion}

Die Arbeit behandelt Fahrzeuge des SAE-Levels 4 und 5. Denn nur diese sind in der Lage, sich ohne Supervision eines Fahrers im Straßenverkehr zu bewegen. 

\subsection{Vorteile}

Autonomes Fahren bringt viele Vorteile mit sich. Durch die unterschiedlichen Sensoren kann ein weitaus größeres Sichtfeld erzeugt werden als es mit dem menschlichen Auge möglich ist. Software trifft Entscheidung binnen Millisekunden und kann akkurate Berechnungen über den Fahrtverlauf durchführen. 
Da es sich wie bereits erwähnt bei 90\% der Unfälle um menschliches Versagen handelt, kann die Einführung autonomer Fahrzeuge die Unfallrate dementsprechend senken \cite{roadSafty}.\\

Durch die effiziente Kommunikation der Fahrzeuge untereinander können Staus vermieden werden. Der Verkehr würde sich besser auf unterschiedliche Teile des Verkehrsnetzes verteilen. Das verkürzt die Fahrzeiten. Ebenso bietet die intelligente Vernetzung der Fahrzeuge untereinander den Vorteil, Folgeunfälle zu verrringern und sie tragen somit auch in diesem Punkt zur Sicherheit bei.\\

Mit geteilten autonomen Fahrzeugen, die Teil der öffentlichen Verkehrsmittel werden würden, könnte die Anzahl der Fahrzeuge auf den Straßen verringert werden. Da die meiste Zeit private Fahrzeuge auf Parkplätzen stehen, würde sich so nicht nur die Parkplatzsuche v. a. in Großstädten angenehmer gestalten. Noch viel wichtiger ist, dass in urbanen Ballungsräumen mehr Freiflächen zur Verfügung stehen würden, die u. a. für den drigend notwendigen Wohnungsbau genutzt werden könnten. Eine Studie der Berryls Strategy Advisor zeigt, dass der Einsatz von 18.000 Robotertaxis den gesamten privaten Verkehr sowie 20\% des Pendlerverkehrs der Stadt München abdecken kann \cite{advisors2017simulation}.\\

Verkürzte Fahrtzeiten und weniger Fahrzeuge auf den Straßen wirken sich ebenso positiv auf die Treibstoffemissionen aus. Somit würde der Einsatz von autonomen Fahrzeugen auch langfristig einen Beitrag zum Umweltschutz leisten.\\

\subsection{Herausforderung}

Die flächendeckende Einführungen von autonomen Fahrzeugen wird die Gesellschaft und auch Gesetzgeber vor Herausforderungen stellen. Zur Zeit existieren nur vage Richtlinien für das Testen und den Betrieb \cite{doi:10.1080/01441647.2018.1494640}. Fraglich ist auch, wie sich autonome Fahrzeuge verhalten sollen, wenn sie Staatengrenzen überqueren. Gerade in Europa sind die meisten Grenzen offen und viele Menschen pendeln zwischen Ländern. Unterschiedliche Gesetzgebungen müssen somit bei der Entwicklung autonomer Fahrzeuge beachtet werden.\\

Gerade in Deutschland, wo ein Auto als Statussymbol gilt, wird sich die Einführung autonomer Fahrzeuge im privaten Sektor als schwierig erweisen. Außerdem sind viele Bewohner europäischer Staaten eher misstrauisch gegenüber neuen Technologien \cite{technikRadar2019}. Bis ein selbstfahrendes Auto von der Mehrheit der Bevölkerung akzeptiert wird, wird es also wohl eine Zeit dauern. Ähnliche Hürden wurden auch in der Publikation \cite{roadblocks} von \citeauthor{roadblocks} angenommen.\\

Ein besonderes Augenmerk muss auf die Entscheidungsfindung in Extremsituation gelegt werden. Als Extremsituation wird eine Konstalltion definiert, in denen ein Unfall unausweichlich ist. Beispielsweise wäre das bei einem plötzlichen mechanischen Versagen der Bremsen der Fall. Fahrzeuge müssen in einer solchen Lage den akzeptabelsten Ausgang finden. In der nächsten Sektion \ref{ssec:entscheidungen-in-extremsituationen} wird explizit dieser Spezialfall näher erläutert. Die Studie \cite{socialDilemma} geht gesondert auf diese Thematik ein. Sie zeigt, dass unterschiedliche Implementierungsmethoden für die Entscheidungsfindung in Extremsituationen das Kaufverhalten und somit auch die Zeit, bis autonome Fahrzeuge gesellschaftlich akzeptiert werden, beeinflussen. In \citeauthor{socialDilemma} wird ebenfalls dargelegt, dass nicht nur die gesellschaftliche Akzeptanz, sondern auch die Regulierung durch Gesetzgeber diesen Prozess noch weiter verlangsamen kann.\\


\subsection{Entscheidungen in Extremsituationen}
\label{ssec:entscheidungen-in-extremsituationen}

Wie werden Entscheidungen autonomer Autos in Gefahrenlagen getroffen? Diese Frage beschäftigt Ethikwissenschaftler. Über die Internetseite \textit{The Moral Machine} wurden weltweit große Mengen an Daten erhoben. Besucher der Internetseite wurden befragt, welcher der ethisch vertretbarere Ausgang einer Extremsituation ist. Dabei mussten Teilnehmer jeweils zwischen zwei Ausgängen wählen. Man hatte beispielsweise die Wahl, das Auto in eine Barriere fahren zu lassen und somit die Insassen zu töten oder Menschen, die rechtens oder auch unrechtens die Ampel überqueren, zu töten. Dabei wurden verschiedene Szenarien mit unterschiedlichen Menschengruppierungen als Insassen und Passanten gezeigt. Die gesammelten Daten stehen frei zur Verfügung.\\

Diese wurden umfassend ausgewertet, um einen Einblick zu haben, wie der allgemeine Status Quo der Gesellschaft zu dieser Thematik ist. Eine Studie hat ergeben, dass die meisten Entscheidungen zu Gunsten größerer Gruppen, junger Personen und Menschen gegenüber Tieren (auch Tiere werden in der \textit{Moral Machine} Internetseite gezeigt) gefallen sind. Jedoch gibt es kulturelle Unterschiede. Beispielsweise werden in östlichen Kulturen, wie etwa in vielen asiatischen Ländern, alte Menschen bevorzugt. Es wurden drei große Hauptstränge gefunden, die die größten Meinungsunterschiede aufweisen. Es handelt sich um die westliche, südliche und östliche Gruppierung \cite{moralMachine}.\\

Auf der Basis der gesammelten Daten wurde ein Algorithmus \cite{votingBasedSystem} entwickelt. Dieser ist in der Lage, in Extremsituation Entscheidungen zu fällen. Es ist jedoch fraglich, wie die Entscheidungen zustande kommen. Europa ist ein sehr diverser Kontinent mit vielen unterschiedlichen Kulturen, Religionen und Sprachen. Die Unterschiede in der Urteilssfindung weisen weitaus größere Unterschiede auf als in den komplett englischsprachigen USA. Laut der Studie befindet sich Frankreich in der südlichen Gruppe, wobei sich Deutschland in der westlichen Gruppierung befindet. Das stellt Entscheidungsträger vor die Herausforderung, ein einheitliches System zu finden, das jedem Land gerecht wird.\\

Deswegen weisen die Autoren des Algorithmus, \citeauthor{votingBasedSystem}, darauf hin, dass die gesammelten Daten nicht komplett sind. Befindet man sich in einer Extremsituation mit seinem Fahrzeug, ergeben sich mehr Optionen als das Fahrzeug in eine Barriere fahren zu lassen oder nicht einzugreifen. Außerdem kann es sein, dass sich geliebte Personen oder Verwandte mit im Fahrzeug befinden. Das hätte Einfluss auf die Entscheidungsfindung bei Menschen. Das sind Faktoren, die beim Sammeln der Daten mit \textit{The Moral Machine} nicht berücksichtigt worden sind.\\

Das Europäische Parlament hat im April 2019 die \textit{Ethik-Leitlinien für eine Vertrauenswürde KI} für den Umgang mit Systemen der Künstlichen Intelligenz veröffentlicht \cite{ec2019ethics}. Autonome Fahrzeuge setzen auf viele Systeme der Künstlichen Intelligenz. Beispiele hierfür sind die Bilder- und Objekterkennung oder die Algorithmen über die Entscheidungsfindung. Bei autonomen Fahrzeugen kommt es jedoch zu einem Konflikt mit diesen Leitlinien. Beispielsweise mit Leitlinie Nr. 1, \textit{Vorrang menschlichen Handelns und menschliche Aufsicht}, die bei Fahrerlosen Autos des SAE-Levels 4 oder 5 nicht zutrifft. Die Autoren des Dokuments, die \citeauthor{ec2019ethics}, kommentieren jedoch, dass die Leitlinien nicht endgültig festgelegt sind. Außerdem müssen diese für bestimmte Teilgebiete der KI angepasst und erweitert werden. Das Dokument ist außerdem in Bearbeitung und soll zukünftig als Grundlage für weitere Leitlinien dienen.\\